\documentclass[11pt]{book}
\usepackage{geometry}
\geometry{reset}

\usepackage[T1]{fontenc}
\usepackage[frenchb]{babel}
\usepackage[utf8x]{inputenc}
\usepackage[dvips,final]{graphicx}
\usepackage{lmodern,array,oldgerm,multirow,amsmath,wrapfig,multicol,exscale,color,colortbl}
\usepackage{vmargin,fancybox,fancyhdr,lastpage}
\usepackage{rotating}
\usepackage{lscape}
% Permet de tourner une page en particulier avec la commande \begin{landscape} … \end{landscape}
\usepackage{arydshln,amssymb}
\usepackage[usenames,dvipsnames]{xcolor}
\usepackage{tikz}
\usepackage{longtable}
\usepackage{pdflscape}
\usepackage{ifoddpage}
\pagestyle{empty}
% Configuration des en-têtes pour le \mainmatter
\fancypagestyle{mainmatterstyle}{
  \fancyhf{}
  \fancyhead[LE]{\slshape Julien Freyermuth \hfill \thepage} % Nom de l'auteur à gauche et numéro de page à droite sur les pages paires
  \fancyhead[RO]{\thepage \hfill \slshape L'Illusion de la Loyauté} % Numéro de page à gauche et titre du livre à droite sur les pages impaires
  \renewcommand{\headrulewidth}{0.4pt} % Ligne sous l'en-tête
}

% Style pour les 3 dernières pages
\fancypagestyle{endmatterstyle}{
  \fancyhf{} % Supprime les en-têtes et pieds de page
  \renewcommand{\headrulewidth}{0pt} % Supprime la ligne sous l'en-tête
}

\newcommand{\checkoddpageandaddblank}{
  \clearpage
  \ifoddpage
    % Si la page est impaire, ne rien faire
  \else
    % Si la page est paire, ajouter une page blanche
    \null
    \clearpage
  \fi
}


\renewcommand{\headrulewidth}{0.4pt}
\renewcommand{\footrulewidth}{0pt}

\geometry{reset}
\geometry{
    paperwidth=12cm,
    paperheight=19cm,
    inner=1.5cm,
    outer=1cm,
    top=1.5cm,
    bottom=2cm
}

\setlength\parindent{0pt}

\begin{document}

\frontmatter

% Page de garde
\begin{titlepage}
    \vspace*{\stretch{1}}
    \begin{center}
        \textbf{\huge L'illusion de la loyauté}\\[0.5cm]
        \textbf{\large Un livre dont vous êtes le héros}\\[2cm]
        % \textbf{\large Julien Freyermuth}
    \end{center}
    \vspace*{\stretch{2}}
\end{titlepage}

\begin{center}
    \textbf{\Large{Règles du jeu}}\\
\end{center}
Vous trouverez au début de ce livre une \textit{Feuille d'Aventure} sur laquelle vous inscrirez
tous les détails de votre quête. Il est conseillé d'en faire des photocopies qui vous permettront de
jouer plusieurs fois.\\

Au cours de votre jeunesse vous avez acquis une HABILETÉ au combat ainsi qu'une ENDURANCE.
Avant d'entamer votre quête, il vous faudra mesurer exactement les dons que vous ont fait la
nature. À cet effet, vous devrez placer devant vous la \textit{Table de Hasard} qui se trouve
à la fin du livre, fermer les yeux et pointer l'extrémité non taillée d'un crayon sur l'un
des chiffres de la \textit{Table} en laissant faire le hasard. Si vous désignez le chiffre
0, vous n'obtenez aucun point.\\

Le premier chiffre que votre crayon aura montré sur la \textit{Table de
Hasard} représentera votre HABILETÉ au combat. Ajoutez 10 à ce
chiffre et inscrivez le total obtenu dans la case HABILETÉ de votre
\textit{Feuille d'Aventure} (si par exemple votre crayon indique le chiffre
4 sur la \textit{Table de Hasard}, vos points d'HABILETÉ seront de 14).
Lorsque vous aurez à combattre, il faudra mesurer votre
HABILETÉ à celle de votre adversaire. Il est donc souhaitable que
votre total d'HABILETÉ soit le plus élevé possible.\\

Le second chiffre que vous désignerez sur la \textit{Table de Hasard} représentera
votre capacité d'ENDURANCE. Ajoutez 20 à ce chiffre et inscrivez le total obtenu
dans la case ENDURANCE de votre \textit{Feuille d'Aventure}. (À titre d'exemple,
si votre crayon indique le chiffre 6 sur la \textit{Table de Hasard}, le total dernières
vos points d'ENDURANCE sera de 26.)\\

Si vous êtes blessé lors d'un combat, vous perdrez des points d'ENDURANCE et si
jamais votre ENDURANCE tombe à zéro, vous saurez alors que vous venez d'être tué et
que votre aventure est terminée. Au cours de votre mission, vous aurez la possibilité
de récupérer des points d'ENDURANCE mais votre total d'ENDURANCE ne pourra en aucun
cas dépasser celui dont vous disposiez au départ de votre mission.

% Si vous avez mené avec succès l'une ou l'autre des précédentes aventures de la série,
% vos points d'HABILETÉ et d'ENDURANCE vous sont déjà connus et c'est avec ces mêmes
% éléments que vous entreprendrez le présent volume. Vous pourrez aussi emporter
% dans cette nouvelle aventure des armes et les objets qui se trouvaient en votre possession
% à la fin de votre précédente mission~: vous devez alors les inscrire en détail
% sur votre \textit{Feuille d'Aventure}. (Rappelez-vous cependant que vous n'avez
% toujours pas droit à plus de deux armes et huit objets dans votre Sac à Dos.)

\begin{center}
  ÉQUIPEMENT
\end{center}

Au départ de votre aventure, vous êtes vêtu d'une tunique noire. Le matériel
dont vous disposez pour assurer votre survie vous seras donné dans la suite de
votre aventure. Vous disposez également, accrochée à votre ceinture, d'une Bourse
de Cuir qui contient des Pièces d'Or. Pour savoir combien de pièces sont en votre possession,
utilisez la \textit{Table de Hasard} à la manière qui vous a été expliquée précédemment~;
le chiffre que vous obtiendrez représentera le nombre de Couronnes (c'est le nom de la monnaie
en cours dans ces régions) dont vous disposez au début de votre mission (inscrivez ce nombre
dans la case Pièces d'Or de votre \textit{Feuille d'Aventure}).\\

\begin{center}
  \textbf{Répartition de l'équipement}
\end{center}

À présent que vous disposez de votre équipement, il vous faut savoir comment il se
répartit afin que vous puissiez le transporter plus aisément. Il n'est pas nécessaire
de prendre des notes à ce sujet, vous vous contenterez, en cas de besoin, de vous référer
aux indications données.

\begin{center}
  \textbf{Combien d'objets pouvez-vous transporter~?}\\
  \textit{Armes}
\end{center}
Vous ne pouvez pas emporter plus de deux armes.

\begin{center}
  \textit{Objets contenus dans votre Sac à Dos}
\end{center}
Ils doivent obligatoirement y être rangé mais la place y est comptée et vous ne pouvez
y transporter que huit objets (Repas inclus).

\begin{center}
  \textit{Objets spéciaux}
\end{center}
Les objets spéciaux ne sont pas rangés dans votre Sac à Dos. Lorsque vous aurez
la possibilité d'acquérir l'un de ces objets spéciaux, il vous sera indiqué de
quelle manière il convient de le transporter.

\begin{center}
  \textit{Pièces d'Or}
\end{center}
Elles sont toujours rangées dans la bourse attachée à votre ceinture. Cette bourse
ne peut pas contenir plus de cinquante Pièces d'Or en tout.

\begin{center}
  \textit{Nourriture}
\end{center}
Vous transportez votre nourriture dans votre Sac à Dos et chaque Repas compte pour
un objet. Tous les objets qui peuvent vous être utiles et que vous aurez la possibilité
d'acquérir au cours de votre aventure sont indiqués avec leur initiale en lettre capitale.
Chaque fois que vous déciderez d'emporter l'un de ces objets avec vous, il vous
faudra l'inscrire sur votre \textit{Feuille d'Aventure}. Ces acquisitions seront
rangées dans votre Sac à Dos sauf s'il vous est spécifié qu'il s'agit d'un Objet Spécial.

\begin{center}
  \textbf{Comment utiliser votre équipement~?}\\
  \textit{Armes}
\end{center}
Les armes vous aident à combattre vos ennemis. Si vous découvrez une arme au cours
de votre aventure, vous pouvez la garder et l'utiliser, mais rappelez-vous que vous
n'avez pas le droit de posséder plus de deux armes à la fois.

\begin{center}
  \textit{Objets contenus dans votre Sac à Dos}
\end{center}
Au cours de votre quête, vous découvrirez divers objets qui pourraient se révéler
utiles et que vous souhaiterez peut-être conserver (rappelez-vous que vous ne pouvez
transporter que huit objets dans votre Sac à Dos). Vous avez le droit à tout moment
d'échanger l'un de ces objets ou tout simplement de vous en débarrasser, mais il vous
est interdit de le faire lorsque vous êtes angagé dans un combat.

\begin{center}
  \textit{Objets spéciaux}
\end{center}
Chacun de ces objets possède des propriétés bien particulières. Parfois, ces
propriétés vous seront décrites au moment même de la découverte de l'objet, d'autres
fois, il vous faudra attendre qu'elles se révèlent au cours de votre aventure. Pour
vous aider à trouver votre chemin, vous disposez d'une carte en début de livre.

\begin{center}
  \textit{Pièces d'Or}
\end{center}
La monnaie en cours dans le royaume est la Couronne qui se présente sous la forme
d'une petite pièce d'or. Au cours de votre aventure, vous pourrez utiliser ces
Couronnes pour vots frais de transport ou de nourriture, mais également, si besoin
est, pour corrompre certains personnages peu scrupuleux. Nombre de créatures que vous
serez amené à rencontrer possèdent des Couronnes d'or. Chaque fois que vous tuerez
l'une de ces créatures, vous aurez le droit de vous emparer de ses pièces d'or et de
les conserver dans votre bourse.

\begin{center}
  \textit{Nourriture}
\end{center}
Tout au long de votre quête, vous aurrez besoin de vous nourrir à intervalles
réguliers. S'il ne vous reste plus de vivres lorsque vous serez dans l'obligation
de prendre un repas, vous perdrez 3 points d'ENDURANCE.

\begin{center}
  \textit{Potion de Guérison}
\end{center}
Elle vous rend des points d'ENDURANCE lorsque vous la buvez après un combat.
Si au cours de votre aventure, vous venez à découvrir d'autres potions, leurs
effets vous seront indiqués en temps utile. Toutes les Potions de Guérison
doivent être rangées dans votre Sac à Dos.

\begin{center}
  \textbf{Règles de combat}
\end{center}
Au cours de votre mission, vous aurez parfois à combattre un ennemi. Le texte vous
précisera en chaque circonstance quels sont les points d'HABILETÉ et d'ENDURANCE de
l'ennemi en question. Vous devrez alors vous efforcer de tuer votre adversaire en
réduisant à zéro les points d'ENDURANCE de ce dernier, tout en essayant de perdre
le moins possible de points d'ENDURANCE au cours de l'affrontement. Au début de
chaque combat, inscrivez votre total d'ENDURANCE, ainsi que celui de l'ennemi sur
votre \textit{Feuille d'Aventure}. Ces indications devront être portées dans la case
Compte Rendu des Combats.\\

Chaque affrontement se déroule de la manière suivante~:
\begin{enumerate}
  % \item Ajoutez à votre total d'HABILETÉ les points supplémentaires
  \item Soustrayez du total de vos points d'HABILETÉ, les points d'HABILETÉ de
  votre adversaire. Le résultat de cette soustraction vous donnera un
  \textit{Quotient d'Attaque}. Inscrivez-le sur votre \textit{Feuille d'Aventure}.
\end{enumerate}

\begin{center}
  \textit{Exemple}
\end{center}
Imaginons que vous disposez d'un total d'HABILETÉ de 15 et que vous êtes attaquée
par un Diable Volant dont les points d'HABILETÉ s'élèvent à 20~: Vous n'avez aucune
possibilité de fuite, il vous faut combattre la créature qui fond sur vous.



\mainmatter
\pagestyle{mainmatterstyle}

test

\newpage

test2

\newpage

\pagestyle{endmatterstyle}
\begin{center}
  \textbf{\Large Table de hasard}\\[1.6cm]
  {
    \renewcommand\arraystretch{2.6}
    \begin{tabular}{|c|c|c|c|c|c|c|c|c|c|}
      \hline
      \huge\textbf{5} & \huge\textbf{3} & \huge\textbf{8} & \huge\textbf{6} & \huge\textbf{7} & \huge\textbf{1} & \huge\textbf{4} & \huge\textbf{2} & \huge\textbf{9} & \huge\textbf{0} \\ \hline
      \huge\textbf{2} & \huge\textbf{0} & \huge\textbf{7} & \huge\textbf{5} & \huge\textbf{3} & \huge\textbf{8} & \huge\textbf{9} & \huge\textbf{1} & \huge\textbf{6} & \huge\textbf{4} \\ \hline
      \huge\textbf{9} & \huge\textbf{6} & \huge\textbf{4} & \huge\textbf{2} & \huge\textbf{0} & \huge\textbf{5} & \huge\textbf{3} & \huge\textbf{7} & \huge\textbf{8} & \huge\textbf{1} \\ \hline
      \huge\textbf{3} & \huge\textbf{7} & \huge\textbf{1} & \huge\textbf{9} & \huge\textbf{4} & \huge\textbf{6} & \huge\textbf{0} & \huge\textbf{5} & \huge\textbf{2} & \huge\textbf{8} \\ \hline
      \huge\textbf{0} & \huge\textbf{2} & \huge\textbf{6} & \huge\textbf{3} & \huge\textbf{9} & \huge\textbf{4} & \huge\textbf{7} & \huge\textbf{8} & \huge\textbf{1} & \huge\textbf{5} \\ \hline
      \huge\textbf{8} & \huge\textbf{1} & \huge\textbf{9} & \huge\textbf{0} & \huge\textbf{5} & \huge\textbf{2} & \huge\textbf{6} & \huge\textbf{3} & \huge\textbf{4} & \huge\textbf{7} \\ \hline
      \huge\textbf{4} & \huge\textbf{9} & \huge\textbf{3} & \huge\textbf{7} & \huge\textbf{1} & \huge\textbf{0} & \huge\textbf{8} & \huge\textbf{6} & \huge\textbf{5} & \huge\textbf{2} \\ \hline
      \huge\textbf{1} & \huge\textbf{5} & \huge\textbf{2} & \huge\textbf{4} & \huge\textbf{8} & \huge\textbf{9} & \huge\textbf{3} & \huge\textbf{0} & \huge\textbf{7} & \huge\textbf{6} \\ \hline
      \huge\textbf{7} & \huge\textbf{8} & \huge\textbf{0} & \huge\textbf{1} & \huge\textbf{6} & \huge\textbf{3} & \huge\textbf{2} & \huge\textbf{9} & \huge\textbf{5} & \huge\textbf{4} \\ \hline
      \huge\textbf{6} & \huge\textbf{4} & \huge\textbf{5} & \huge\textbf{8} & \huge\textbf{2} & \huge\textbf{7} & \huge\textbf{1} & \huge\textbf{4} & \huge\textbf{0} & \huge\textbf{9} \\ \hline
    \end{tabular}
  }
\end{center}
    

\clearpage

\checkoddpageandaddblank

\begin{flushright}
    \textbf{\Large Table des}\\
\end{flushright}

Quotient d'attaque\\

\begin{minipage}[t]{\textwidth}
  \begin{flushright}
    \scalebox{0.8}{\renewcommand\arraystretch{1.6}\begin{tabular}{c|c|c c|c c|c c|c c|c c|c c|}
        \cline{3-14}
        \multicolumn{2}{c|}{}& \multicolumn{2}{c|}{$\leq$ -11}&
            \multicolumn{2}{c|}{-10/-9}&
            \multicolumn{2}{c|}{-8/-7}&
            \multicolumn{2}{c|}{-6/-5}&
            \multicolumn{2}{c|}{-4/-3}&
            \multicolumn{2}{c|}{-2/-1}\\
        \cline{2-14}

        \multirow{20}{*}{\rotatebox{90}{Chiffre donné par la table de hasard}}& \multirow{2}{*}{1}& E& -0& E& -0& E& -0& E& -0& E& -1& E& -2\\
        \cline{3-14}
        & & V& T& V& T& V&  -8& V&  -6& V&  -6& V&  -5\\
        \cline{2-14}
        
        & & E&  -0& E&  -0& E&  -0& E&  -1& E&  -2& E&  -3\\
        \cline{3-14}
        & \multirow{-2}{*}{2}& V& T& V&  -8& V&  -7& V&  -6& V&  -5& V&  -5\\
        \cline{2-14}

        & \multirow{2}{*}{3}& E&  -0& E&  -0& E&  -1& E&  -2& E&  -3& E&  -4\\
        \cline{3-14}
        & & V&  -8& V&  -7& V&  -6& V&  -5& V&  -5& V&  -4\\
        \cline{2-14}

        & \multirow{2}{*}{4}& E&  -0& E&  -1& E&  -2& E&  -3& E&  -4& E&  -5\\
        \cline{3-14}
        & & V&  -8& V&  -7& V&  -6& V&  -5& V&  -4& V&  -4\\
        \cline{2-14}

        & \multirow{2}{*}{5}& E&  -1& E&  -2& E&  -3& E&  -4& E&  -5& E&  -6\\
        \cline{3-14}
        & & V&  -7& V&  -6& V&  -5& V&  -4& V&  -4& V&  -3\\
        \cline{2-14}

        & \multirow{2}{*}{6}& E&  -2& E&  -3& E&  -4& E&  -5& E&  -6& E&  -7\\
        \cline{3-14}
        & & V&  -6& V&  -6& V&  -5& V&  -4& V&  -3& V&  -2\\
        \cline{2-14}

        & \multirow{2}{*}{7}& E&  -3& E&  -4& E&  -5& E&  -6& E&  -7& E&  -8\\
        \cline{3-14}
        & & V&  -5& V&  -5& V&  -4& V&  -3& V&  -2& V&  -2\\
        \cline{2-14}

        & \multirow{2}{*}{8}& E&  -4& E&  -5& E&  -6& E&  -7& E&  -8& E&  -9\\
        \cline{3-14}
        & & V&  -4& V&  -4& V&  -3& V&  -2& V&  -1& V&  -1\\
        \cline{2-14}

        & \multirow{2}{*}{9}& E&  -5& E&  -6& E&  -7& E&  -8& E&  -9& E& -10\\
        \cline{3-14}
        & & V&  -3& V&  -3& V&  -2& V& 0& V& 0& V& 0\\
        \cline{2-14}

        & \multirow{2}{*}{0}& E&  -6& E&  -7& E&  -8& E&  -9& E& -10& E& -11\\
        \cline{3-14}
        & & V& 0& V& 0& V& 0& V& 0& V& 0& V& 0\\
        \cline{2-14}
    \end{tabular}}
  \end{flushright}
\end{minipage}

\begin{center}
    E = Ennemi \quad\quad V = Vous
\end{center}

\newpage

\begin{flushleft}
    \textbf{\Large coups portée}\\
\end{flushleft}

\textcolor{white}{Quotient d'attaque}\\

\begin{minipage}[t]{\textwidth}
    \scalebox{0.8}{\renewcommand\arraystretch{1.6}\begin{tabular}{|c c|c c|c c|c c|c c|c c|c c|c|}
        \cline{1-14}
        \multicolumn{2}{|c|}{0/0}&
            \multicolumn{2}{c|}{+1/+2}&
            \multicolumn{2}{c|}{+3/+4}&
            \multicolumn{2}{c|}{+5/+6}&
            \multicolumn{2}{c|}{+7/+8}&
            \multicolumn{2}{c|}{+9/+10}&
            \multicolumn{2}{c|}{$\geq$+11}\\
        \hline

        E& -3& E& -4& E& -5& E& -6& E& -7& E& -8& E& -9& \multirow{2}{*}{1}\\
        \cline{1-14}
        V& -5& V& -5& V& -4& V& -4& V& -4& V& -3& V& -3& \\
        \hline
        
        E& -4& E& -5& E& -6& E& -7& E& -8& E& -9& E& -10& \multirow{2}{*}{2}\\
        \cline{1-14}
        V& -4& V& -4& V& -3& V& -3& V& -3& V& -3& V& -2& \\
        \hline
        
        E& -5& E& -6& E& -7& E& -8& E& -9& E& -10& E& -11& \multirow{2}{*}{3}\\
        \cline{1-14}
        V& -4& V& -3& V& -3& V& -3& V& -2& V& -2& V& -2& \\
        \hline
        
        E& -6& E& -7& E& -8& E& -9& E& -10& E& -11& E& -12& \multirow{2}{*}{4}\\
        \cline{1-14}
        V& -3& V& -3& V& -2& V& -2& V& -2& V& -2& V& -2& \\
        \hline
        
        E& -7& E& -8& E& -9& E& -10& E& -11& E& -12& E& -14& \multirow{2}{*}{5}\\
        \cline{1-14}
        V& -2& V& -2& V& -2& V& -2& V& -2& V& -2& V& -1& \\
        \hline
        
        E& -8& E& -9& E& -10& E& -11& E& -12& E& -14& E& -16& \multirow{2}{*}{6}\\
        \cline{1-14}
        V& -2& V& -2& V& -2& V& -1& V& -1& V& -1& V& -1& \\
        \hline
        
        E& -9& E& -10& E& -11& E& -12& E& -14& E& -16& E& -18& \multirow{2}{*}{7}\\
        \cline{1-14}
        V& -1& V& -1& V& -1& V& -0& V& -0& V& -0& V& -0& \\
        \hline
        
        E& -10& E& -11& E& -12& E& -14& E& -16& E& -18& E& T& \multirow{2}{*}{8}\\
        \cline{1-14}
        V& -0& V& -0& V& -0& V& -0& V& -0& V& -0& V& -0& \\
        \hline
        
        E& -11& E& -12& E& -14& E& -16& E& -18& E& T& E& T& \multirow{2}{*}{9}\\
        \cline{1-14}
        V& -0& V& -0& V& -0& V& -0& V& -0& V& -0& V& -0& \\
        \hline
        
        E& -12& E& -14& E& -16& E& -18& E& T& E& T& E& T& \multirow{2}{*}{0}\\
        \cline{1-14}
        V& -0& V& -0& V& -0& V& -0& V& -0& V& -0& V& -0& \\
        \hline
    \end{tabular}}
\end{minipage}

\begin{center}
    T = Tué sur le coup
\end{center}


\end{document}
