% \documentclass[a4paper,10pt]{article}
\documentclass[10pt]{book}
\usepackage{geometry}
\geometry{reset}

\usepackage[T1]{fontenc}
\usepackage[frenchb]{babel}
\usepackage[utf8x]{inputenc}
\usepackage[dvips,final]{graphicx}
\usepackage{lmodern,array,oldgerm,multirow,amsmath,wrapfig,multicol,exscale,color,colortbl}
\usepackage{vmargin,fancybox,fancyhdr,lastpage}
\usepackage{rotating}
\usepackage{lscape}
% Permet de tourner une page en particulier avec la commande \begin{landscape} … \end{landscape}
\usepackage{arydshln,amssymb}
\usepackage[usenames,dvipsnames]{xcolor}
\usepackage{tikz}
\usepackage{longtable}
\usepackage{pdflscape}
\usepackage{ifoddpage}
\pagestyle{empty}
% Configuration des en-têtes pour le \mainmatter
\fancypagestyle{mainmatterstyle}{
  \fancyhf{}
  \fancyhead[LE]{\slshape Julien Freyermuth \hfill \thepage} % Nom de l'auteur à gauche et numéro de page à droite sur les pages paires
  \fancyhead[RO]{\thepage \hfill \slshape L'Illusion de la Loyauté} % Numéro de page à gauche et titre du livre à droite sur les pages impaires
  \renewcommand{\headrulewidth}{0.4pt} % Ligne sous l'en-tête
}

% Style pour les 3 dernières pages
\fancypagestyle{endmatterstyle}{
  \fancyhf{} % Supprime les en-têtes et pieds de page
  \renewcommand{\headrulewidth}{0pt} % Supprime la ligne sous l'en-tête
}

\newcommand{\checkoddpageandaddblank}{
  \clearpage
  \ifoddpage
    % Si la page est impaire, ne rien faire
  \else
    % Si la page est paire, ajouter une page blanche
    \null
    \clearpage
  \fi
}


\renewcommand{\headrulewidth}{0.4pt}
\renewcommand{\footrulewidth}{0pt}

\geometry{reset}
\geometry{
    paperwidth=12cm,
    paperheight=19cm,
    inner=1.5cm,
    outer=1cm,
    top=1.5cm,
    bottom=2cm
}

\setlength\parindent{0pt}

\begin{document}

\frontmatter

% Page de garde
\begin{titlepage}
    \vspace*{\stretch{1}}
    \begin{center}
        \textbf{\huge L'illusion de la loyauté}\\[0.5cm]
        \textbf{\large Un livre dont vous êtes le héros}\\[2cm]
        % \textbf{\large Julien Freyermuth}
    \end{center}
    \vspace*{\stretch{2}}
\end{titlepage}

\begin{center}
    \textbf{\Large{Règles du jeu}}\\
\end{center}
Vous trouverez au début de ce livre une \textit{Feuille d'Aventure} sur laquelle vous inscrirez
tous les détails de votre quête. Il est conseillé d'en faire des photocopies qui vous permettront de
jouer plusieurs fois.\\

Au cours de votre jeunesse vous avez acquis une HABILETÉ au combat ainsi qu'une ENDURANCE.
Avant d'entamer votre quête, il vous faudra mesurer exactement les dons que vous ont fait la
nature. À cet effet, vous devrez placer devant vous la \textit{Table de Hasard} qui se trouve
à la fin du livre, fermer les yeux et pointer l'extrémité non taillée d'un crayon sur l'un
des chiffres de la \textit{Table} en laissant faire le hasard. Si vous désignez le chiffre
0, vous n'obtenez aucun point.\\

Le premier chiffre que votre crayon aura montré sur la \textit{Table de
Hasard} représentera votre HABILETÉ au combat. Ajoutez 10 à ce
chiffre et inscrivez le total obtenu dans la case HABILETÉ de votre
\textit{Feuille d'Aventure} (si par exemple votre crayon indique le chiffre
4 sur la \textit{Table de Hasard}, vos points d'HABILETÉ seront de 14).
Lorsque vous aurez à combattre, il faudra mesurer votre
HABILETÉ à celle de votre adversaire. Il est donc souhaitable que
votre total d'HABILETÉ soit le plus élevé possible.\\


\mainmatter
\pagestyle{mainmatterstyle}

test

\newpage

test2

\newpage

\pagestyle{endmatterstyle}
\begin{center}
    \textbf{Table de hasard}
\end{center}

\clearpage

\checkoddpageandaddblank

\begin{flushright}
    \textbf{\Large Table des}\\
\end{flushright}

Quotient d'attaque\\

\begin{minipage}[t]{\textwidth}
  \begin{flushright}
    \scalebox{0.8}{\renewcommand\arraystretch{1.6}\begin{tabular}{c|c|c c|c c|c c|c c|c c|c c|}
        \cline{3-14}
        \multicolumn{2}{c|}{}& \multicolumn{2}{c|}{$\leq$ -11}&
            \multicolumn{2}{c|}{-10/-9}&
            \multicolumn{2}{c|}{-8/-7}&
            \multicolumn{2}{c|}{-6/-5}&
            \multicolumn{2}{c|}{-4/-3}&
            \multicolumn{2}{c|}{-2/-1}\\
        \cline{2-14}

        \multirow{20}{*}{\rotatebox{90}{Chiffre donné par la table de hasard}}& \multirow{2}{*}{1}& E& -0& E& -0& E& -0& E& -0& E& -1& E& -2\\
        \cline{3-14}
        & & V& T& V& T& V&  -8& V&  -6& V&  -6& V&  -5\\
        \cline{2-14}
        
        & & E&  -0& E&  -0& E&  -0& E&  -1& E&  -2& E&  -3\\
        \cline{3-14}
        & \multirow{-2}{*}{2}& V& T& V&  -8& V&  -7& V&  -6& V&  -5& V&  -5\\
        \cline{2-14}

        & \multirow{2}{*}{3}& E&  -0& E&  -0& E&  -1& E&  -2& E&  -3& E&  -4\\
        \cline{3-14}
        & & V&  -8& V&  -7& V&  -6& V&  -5& V&  -5& V&  -4\\
        \cline{2-14}

        & \multirow{2}{*}{4}& E&  -0& E&  -1& E&  -2& E&  -3& E&  -4& E&  -5\\
        \cline{3-14}
        & & V&  -8& V&  -7& V&  -6& V&  -5& V&  -4& V&  -4\\
        \cline{2-14}

        & \multirow{2}{*}{5}& E&  -1& E&  -2& E&  -3& E&  -4& E&  -5& E&  -6\\
        \cline{3-14}
        & & V&  -7& V&  -6& V&  -5& V&  -4& V&  -4& V&  -3\\
        \cline{2-14}

        & \multirow{2}{*}{6}& E&  -2& E&  -3& E&  -4& E&  -5& E&  -6& E&  -7\\
        \cline{3-14}
        & & V&  -6& V&  -6& V&  -5& V&  -4& V&  -3& V&  -2\\
        \cline{2-14}

        & \multirow{2}{*}{7}& E&  -3& E&  -4& E&  -5& E&  -6& E&  -7& E&  -8\\
        \cline{3-14}
        & & V&  -5& V&  -5& V&  -4& V&  -3& V&  -2& V&  -2\\
        \cline{2-14}

        & \multirow{2}{*}{8}& E&  -4& E&  -5& E&  -6& E&  -7& E&  -8& E&  -9\\
        \cline{3-14}
        & & V&  -4& V&  -4& V&  -3& V&  -2& V&  -1& V&  -1\\
        \cline{2-14}

        & \multirow{2}{*}{9}& E&  -5& E&  -6& E&  -7& E&  -8& E&  -9& E& -10\\
        \cline{3-14}
        & & V&  -3& V&  -3& V&  -2& V& 0& V& 0& V& 0\\
        \cline{2-14}

        & \multirow{2}{*}{0}& E&  -6& E&  -7& E&  -8& E&  -9& E& -10& E& -11\\
        \cline{3-14}
        & & V& 0& V& 0& V& 0& V& 0& V& 0& V& 0\\
        \cline{2-14}
    \end{tabular}}
  \end{flushright}
\end{minipage}

\begin{center}
    E = Ennemi \quad\quad V = Vous
\end{center}

\newpage

\begin{flushleft}
    \textbf{\Large coups portée}\\
\end{flushleft}

\textcolor{white}{Quotient d'attaque}\\

\begin{minipage}[t]{\textwidth}
    \scalebox{0.8}{\renewcommand\arraystretch{1.6}\begin{tabular}{|c c|c c|c c|c c|c c|c c|c c|c|}
        \cline{1-14}
        \multicolumn{2}{|c|}{0/0}&
            \multicolumn{2}{c|}{+1/+2}&
            \multicolumn{2}{c|}{+3/+4}&
            \multicolumn{2}{c|}{+5/+6}&
            \multicolumn{2}{c|}{+7/+8}&
            \multicolumn{2}{c|}{+9/+10}&
            \multicolumn{2}{c|}{$\geq$+11}\\
        \hline

        E& -3& E& -4& E& -5& E& -6& E& -7& E& -8& E& -9& \multirow{2}{*}{1}\\
        \cline{1-14}
        V& -5& V& -5& V& -4& V& -4& V& -4& V& -3& V& -3& \\
        \hline
        
        E& -4& E& -5& E& -6& E& -7& E& -8& E& -9& E& -10& \multirow{2}{*}{2}\\
        \cline{1-14}
        V& -4& V& -4& V& -3& V& -3& V& -3& V& -3& V& -2& \\
        \hline
        
        E& -5& E& -6& E& -7& E& -8& E& -9& E& -10& E& -11& \multirow{2}{*}{3}\\
        \cline{1-14}
        V& -4& V& -3& V& -3& V& -3& V& -2& V& -2& V& -2& \\
        \hline
        
        E& -6& E& -7& E& -8& E& -9& E& -10& E& -11& E& -12& \multirow{2}{*}{4}\\
        \cline{1-14}
        V& -3& V& -3& V& -2& V& -2& V& -2& V& -2& V& -2& \\
        \hline
        
        E& -7& E& -8& E& -9& E& -10& E& -11& E& -12& E& -14& \multirow{2}{*}{5}\\
        \cline{1-14}
        V& -2& V& -2& V& -2& V& -2& V& -2& V& -2& V& -1& \\
        \hline
        
        E& -8& E& -9& E& -10& E& -11& E& -12& E& -14& E& -16& \multirow{2}{*}{6}\\
        \cline{1-14}
        V& -2& V& -2& V& -2& V& -1& V& -1& V& -1& V& -1& \\
        \hline
        
        E& -9& E& -10& E& -11& E& -12& E& -14& E& -16& E& -18& \multirow{2}{*}{7}\\
        \cline{1-14}
        V& -1& V& -1& V& -1& V& -0& V& -0& V& -0& V& -0& \\
        \hline
        
        E& -10& E& -11& E& -12& E& -14& E& -16& E& -18& E& T& \multirow{2}{*}{8}\\
        \cline{1-14}
        V& -0& V& -0& V& -0& V& -0& V& -0& V& -0& V& -0& \\
        \hline
        
        E& -11& E& -12& E& -14& E& -16& E& -18& E& T& E& T& \multirow{2}{*}{9}\\
        \cline{1-14}
        V& -0& V& -0& V& -0& V& -0& V& -0& V& -0& V& -0& \\
        \hline
        
        E& -12& E& -14& E& -16& E& -18& E& T& E& T& E& T& \multirow{2}{*}{0}\\
        \cline{1-14}
        V& -0& V& -0& V& -0& V& -0& V& -0& V& -0& V& -0& \\
        \hline
    \end{tabular}}
\end{minipage}

\begin{center}
    T = Tué sur le coup
\end{center}


\end{document}
